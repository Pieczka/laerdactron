\section{Введение}

\subsection{электромагнетизм}

Древнегреческие философы знали, что кусок янтаря
потер смог привлечь бит соломы. также
было известно, что некоторые `` stones''found в
природы, ныне известный как магнетит, смогли atrir
железа. Эти два наблюдения формы происхождении наук
электричества и магнетизма. Оба были разработаны
независимо друг от друга до 1820 года, когда Ганс Христиан Эрстед
обнаружили связь между ними: электрический ток,
ходьбе проволоки, вызванных отклонением магнитной стрелки
компас. Из этих наблюдений, мы приходим в возрасте
мы живем электронные (электрон, греческий
$\eta\lambda\epsilon\kappa\rho o\nu$ \textipa{/E:lektron/},
желтым) \cite{halliday}.

\subsection{Электрический заряд}


