\section{Введение}

\subsection{электромагнетизм}

Ещё античные греки знали, что янтарь натёртый о кошачью шерсть
притягивает кусочки соломы. Также им были известны некоторые встречаемые в природе камни, такие как магнетит,
способные притягивать железо. Эти два наблюдения положили начала наукам об электричестве и магнитизме. Обе науки
развивались независимо друг от друга до тех пор как в 1820 году Ганс Христиан Эрстед не обнаружил связь между ними: 
протекающий по проводам ток вызывает отклонение магнитной стрелки компаса. От этих наблюдений и начинается век 
электричества, в котором мы живём и поныне.(электрон, от греческого
$\eta\lambda\epsilon\kappa\rho o\nu$ \textipa{/E:lektron/},
желтым) \cite{halliday}.

\subsection{Электрический заряд}


