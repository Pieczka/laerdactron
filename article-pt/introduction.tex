\section{Introdução}

\subsection{Eletromagnetismo}

Os antigos filósofos gregos sabiam que um pedaço de âmbar
friccionado era capaz de atrair fragmentos de palha. Também
tinham conhecimento de que certas ``pedras'' encontradas na
natureza, hoje conhecidas como magnetitas, eram capazes de atrir
o ferro. Essas duas observações formam a origem das ciências
da eletricidade e do magnetismo. Ambas desenvolveram-se
independentemente até 1820, quando Hans Christian Oersted 
descobriu uma conexão entre elas: uma corrente elétrica, 
percorrendo um fio, ocasionava a deflexão da agulha imantada de
uma bússola. A partir dessas observações, chegamos a era da 
eletrônica em que vivemos (elétron, do grego 
$\eta\lambda\epsilon\kappa\rho o\nu$ \textipa{/E:lektron/}, 
significa âmbar) \cite{halliday}.

\subsection{Carga Elétrica}


